
%% the 2 next lines are helps texstudio set code+language 
% !TeX encoding = UTF-8
% !TeX spellcheck = fr_FR

%% the next line helps texstsudio to set the root document
% !TeX root = ./my_root_doc_name.tex


\documentclass[10pt,a4paper]{article}
\usepackage[utf8]{inputenc}
\usepackage{amsmath}
\usepackage{amsfonts}
\usepackage{amssymb}
\usepackage{graphicx}
\usepackage{paralist}
\usepackage[left=2.00cm, right=2.00cm, top=2.00cm, bottom=2.00cm]{geometry}
\usepackage{listings}
\usepackage{xcolor}
\usepackage{sk_colors_def}
\usepackage{sk_cs_def}
\usepackage{sk_macros}
\usepackage{csquotes}
\usepackage{dirtree}
\usepackage{empheq}
\usepackage{physics}
\usepackage{mathtools}
\usepackage{lastpage}
\usepackage{tcolorbox}
\tcbuselibrary{breakable} %allows to have box spanning multiple pages
\usepackage{todonotes}
\usepackage{algpseudocode} % this loads the package algorithmicx

% french stuff
\usepackage[french]{babel}
\frenchbsetup{og = �, fg = �}

%lineo package and special patch for correcting missing numbering
\usepackage[mathlines]{lineno}
\linenumbers
\newcommand*\patchAmsMathEnvironmentForLineno[1]{%
  \expandafter\let\csname old#1\expandafter\endcsname\csname #1\endcsname
  \expandafter\let\csname oldend#1\expandafter\endcsname\csname end#1\endcsname
  \renewenvironment{#1}%
     {\linenomath\csname old#1\endcsname}%
     {\csname oldend#1\endcsname\endlinenomath}}% 
\newcommand*\patchBothAmsMathEnvironmentsForLineno[1]{%
  \patchAmsMathEnvironmentForLineno{#1}%
  \patchAmsMathEnvironmentForLineno{#1*}}%
\AtBeginDocument{%
\patchBothAmsMathEnvironmentsForLineno{equation}%
\patchBothAmsMathEnvironmentsForLineno{align}%
\patchBothAmsMathEnvironmentsForLineno{flalign}%
\patchBothAmsMathEnvironmentsForLineno{alignat}%
\patchBothAmsMathEnvironmentsForLineno{gather}%
\patchBothAmsMathEnvironmentsForLineno{multline}%
}




\newenvironment{skitemize}{\begin{compactitem}[\hspace{1cm}$\bullet$]}{\end{compactitem}}

\title{template doc}
\date{}
\author{}

\begin{document}

Voici du texte \emphg{bla}


Voici un système d'équations
\begin{subequations}
\begin{empheq}[left=\empheqlbrace]{align}
x &= a, 
\label{eq: system x eq}
\\
z &= f.
\label{eq: system z eq}
\end{empheq}
\label{eq: system}
\end{subequations}

des caractères gras en mode math (avec le package physics):  
$\vb{A}$, $\vb{f}$, gras et italique: $\vb*{A}$, $\vb*{\alpha}$ (pas de gras sans italique pour les lettres grecques)

avec mathbf: $\mathbf{A}$


Des dérivés avec physics.sty:
\begin{align*}
\pdv{f}{x}
\\
\dv{f}{x}
\\
\pdv{f}{x}{y}
\\
\pdv{f}{x}{y}
\end{align*}

Des cases à droite avec mathtools
\begin{equation*}
\begin{rcases*}
x &= a
\\
y &= b
\end{rcases*}
\implies f(x) = 0
\end{equation*}

une list compactitem
\begin{compactitem}
\item bla
\item blabla
\end{compactitem}
\medskip

\begin{compactitem}[$\bullet$]
\item bla
\item blabla
\end{compactitem}

\medskip

Voici une liste imbriquée
\begin{compactitem}[\hspace{1cm}$\bullet$]
\item bla
\item blabla
\begin{compactitem}[\hspace{1cm}$\circ$]
\item bla
\item blabla
\end{compactitem}
\end{compactitem}


\medskip

\framebox{
\begin{minipage}{.4\textwidth}
\begin{center}
Fig.~1~: structure du directory {\tt project\_dir} \\[5pt]
\begin{minipage}{\textwidth}
\dirtree{%
.1 project\_dir.
.2 Makefile.
.2 myHeader.h.
.2 globvar.h.
.2 globvar.c.
.2 file.c.
.2 prog.c.
}
\end{minipage}
\end{center}
\end{minipage}
}

\bigskip

Quelques opérateurs logiques: $\implies$ \quad $\iff$ \quad $\impliedby$


\bigskip

a text with a box
\begin{tcolorbox}[title=My Box, colupper=red,colback=green!50!black,breakable]
This is a box
\end{tcolorbox}


\end{document}
